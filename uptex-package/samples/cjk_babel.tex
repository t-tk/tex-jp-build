%
% test of Babel+UTF8 and CJK multilingual text
%
% This file is originally a test file for Utf82TeX
%%% Utf82TeX sample TeX file for Unix
%%% (c) 2004-2005, Isao YASUDA, isao@yasuda.homeip.net
%%% $Id: utf82tex-sample.tex,v 1.2 2006/12/09 15:04:25 isao Exp $
% and is modified for upLaTeX
%
\documentclass{ujarticle}
%\usepackage{lmodern}% Latin Modern
\usepackage[utf8]{inputenc}
%\usepackage[10pt]{type1ec}
\usepackage[OT2,T2A,T2B,T2C,T1]{fontenc}
\usepackage[french,german,czech,russian,japanese]{babel}

\DeclareFontFamily{JY2}{jpnrm}{}
\DeclareFontFamily{JY2}{schrm}{}
\DeclareFontFamily{JY2}{tchrm}{}
\DeclareFontFamily{JY2}{korrm}{}
\DeclareFontShape{JY2}{jpnrm}{m}{n}{<->s*[0.962216]upjpnrm-h}{}
\DeclareFontShape{JY2}{schrm}{m}{n}{<->s*[0.962216]upschrm-h}{}
\DeclareFontShape{JY2}{tchrm}{m}{n}{<->s*[0.962216]uptchrm-h}{}
\DeclareFontShape{JY2}{korrm}{m}{n}{<->s*[0.962216]upkorrm-h}{}
\DeclareFontShape{JY2}{jpnrm}{bx}{n}{<->ssub*jpnrm/m/n}{}
\DeclareFontShape{JY2}{schrm}{bx}{n}{<->ssub*schrm/m/n}{}
\DeclareFontShape{JY2}{tchrm}{bx}{n}{<->ssub*tchrm/m/n}{}
\DeclareFontShape{JY2}{korrm}{bx}{n}{<->ssub*korrm/m/n}{}
\DeclareRobustCommand\jpnrm{\kanjifamily{jpnrm}\selectfont}
\DeclareRobustCommand\schrm{\kanjifamily{schrm}\selectfont}
\DeclareRobustCommand\tchrm{\kanjifamily{tchrm}\selectfont}
\DeclareRobustCommand\korrm{\kanjifamily{korrm}\selectfont}

\begin{document}
\kcatcode"A7=15% U+A7:§  not cjk character
\kcatcode"C0=15% U+C0:À  not cjk character
\kcatcode`Ŕ=15
\kcatcode`П=15
\kcatcode`“=15

\selectlanguage{french}
\section{フランス語 Français}
Souvent, la main portée au front, debout sur les 
vaisseaux,tandis que la lune se balançait entre 
les mâts d'une façon irrégulière, je me suis surpris, 
faisant abstraction de tout ce qui n'était pas le but 
que je poursuivais, m'efforçant de résoudre ce 
difficile problème!

\hfill {\em Conte de Lautréamont, «Les Chants de Maldoror»}%
\qquad\qquad

\hfill\today


\selectlanguage{german}
\section{ドイツ語 Deutsch}
Dann ließ sie ihre Hände langsam über meine Wangen
heruntergleiten, und ihr Blick ruhte mit unendlicher
Innigkeit auf mir.
Sie schüttelte den Kopf mit einem schmerzlichen Ausdruck,
als könnte sie irgend etwas nicht fassen.
,,Mußst du denn schon heute weg?{}`` 
fragte sie leise.

\hfill {\em A. Schnitzler, «Die Frau des Weisen»}%
\qquad\qquad

\hfill\today


\selectlanguage{czech}%
\section{チェコ語 Czech}
Posláním sdružení je vytvářet předpoklady 
pro všestranné využívání a další rozvoj jazyka 
počítačové typografie \TeX{} a příbuzného programového 
vybavení pro stolní tisk, 
zejména mezi českými a slovenskými uživateli.

\hfill {\em Czechoslovak \TeX{} Users Group}%
\qquad\qquad

\hfill\today


%\selectlanguage{nippon}
\section{Latin-1,Latin-2}
ÄÁÅÂÀÃ ÏÍÎÌ ÜÚÛÙ ËÉÆÊÈ ÖÓØÔÒÕ ~Þ ÝÐÇÑ ¡¿\\
äáåâàã ïíîì üúûù ëéæêè öóøôòõ ßþÿýðçñ\\
ÁĄÂĂ ÍÎ Ú{\selectlanguage{czech}Ů}Ű ÉĘĚ ÓŐÔ 
ŔŘŢŤÝŚŞŠĐĎĹŁĽŹŻŽĆÇČŃŇ\\
áąâă íî ú{\selectlanguage{czech}ů}ű éęě óőô 
ŕřţťýśşšđďĺłľźżžćçčńň

\kcatcode"A7=18% other_kchar
``?`But aren't Kafka's Schlo{\ss} and {\AE}sop's {\OE}uvres
often na{\"\i}ve  vis-\`a-vis the d{\ae}monic ph{\oe}nix's official r\^ole
in fluffy souffl\'es?''

\kcatcode"A7=15% not cjk character
``?`But aren't Kafka's Schlo{\ss} and {\AE}sop's {\OE}uvres
often na{\"\i}ve  vis-\`a-vis the d{\ae}monic ph{\oe}nix's official r\^ole
in fluffy souffl\'es?''

% “:U+201C  ”:U+201D  ’:U+2019
“¿But aren’t Kafka’s Schloß and Æsop’s Œuvres
often naïve vis-à-vis the dæmonic phœnix’s official rôle
in fluffy soufflés?”

\selectlanguage{russian}
\section{ロシア語 \fontencoding{T2A}\selectfont{}Русский}
\fontencoding{T2A}\selectfont
Прежде всего откроем тайну которую Мастер не пожелал
открыть Иванушке.
Возлюбленную его звали Маргаритою Николаевной.
Все, что Мастер говорил о ней, было сущей правдой.
Он описал свою возлюбленную верно.
Она была красива и умна.

\hfill {\em М. Булгаков, «Мастер и Маргарита»}
\qquad\qquad

OT2:\\
\fontencoding{OT2}\selectfont
АБВГДЕЖЗИЙКЛМНОПРСТУФХЦЧШЩЪЫЬЭЮЯ\\
абвгдежзийклмнопрстуфхцчшщъыьэюя

T2A:\\
\fontencoding{T2A}\selectfont
АБВГДЕЖЗИЙКЛМНОПРСТУФХЦЧШЩЪЫЬЭЮЯ\\
абвгдежзийклмнопрстуфхцчшщъыьэюя

\hfill\today

\fontencoding{T1}\selectfont

\selectlanguage{japanese}
\section{日本語}
\jpnrm
雪後庵は起伏の多い小石川界隈の高臺にあつて、幸ひに戰災を免かれた。
三千坪に及ぶ名高い小堀遠洲流の名園と共に、京都のとある名刹から
移された中雀門も、奈良の古い寺をそのまゝ移した玄關や客殿も、
あとに建てられた大廣閒も、何一つ損なはれてゐなかつた。

戰後の財產税さわぎの只中に、雪後庵は元の持主の實業家の茶人の手から、
美しい元氣な女主人の手に渡つて、たちまち名高い料理屋になつた。

\hfill 三島由紀夫『宴のあと』\qquad\qquad

\西暦false
%\def\today{%
% 平成\number\heisei 年
% \number\month 月
% \number\day 日
%}

\hfill\today
\typeout{日本語}

\section{中国語・簡体字 {\schrm 简体中文}}
{\schrm
本常问问答集~(FAQ list)~是从一些经常被问到的问题及其适当的解答中,
以方便的形式摘要而出的。
跟上一版不同的是,其编排结构已彻底改变。
有关新结构的细节,
可参考「如何阅读本问答集及了解其编排结构」该项中的说明。

\def\today{%
 \number\year 年
 \number\month 月
 \number\day 日
}

\hfill\today
}
\typeout{简体中文}

\section{中国語・繁体字 {\tchrm 繁體中文}}
{\tchrm
本常問問答集~(FAQ list)~是從一些經常被問到的問題及其適當的解答中,
以方便的形式摘要而出的。
跟上一版不同的是,其編排結構已徹底改變。
有關新結構的細節,
可參考「如何閱讀本問答集及了解其編排結構」該項中的說明。

\def\today{%
 \number\year 年
 \number\month 月
 \number\day 日
}

\hfill\today
}
\typeout{繁體中文}

\section{韓国語 {\korrm 한국어}}
{\korrm
\xkanjiskip=.1zw plus 1pt minus 1pt
% 
% upTeX treats a linebreak after a hangul character as a space.
% Ref. http://project.ktug.or.kr/omega-cjk/tug2004-preprint.pdf
이 FAQ은 자주 반복되는
질문과 그에 대한 대답을
간단명료한
양식으로%
모아 엮어졌습니다. 
이 FAQ의 구조는 지난 판에 비하여
획기적으로변경되었습니다.
상세한 것은 ``이 FAQ을 어떻게 읽을
것인가'' 라는 대목을 참조하시기 바랍니다.


\def\today{%
 \number\year 연%
 \number\month 월%
 \number\day 일%
}

\hfill\today
}

%% test of xkanjiskip
{\korrm
\xkanjiskip=.25zw plus 1pt minus 1pt

이 FAQ을

\xkanjiskip=5zw plus 1pt minus 1pt

이 FAQ을

\xkanjiskip=-.5zw minus 1pt

이 FAQ을

}

\typeout{한국어}

\jpnrm
\section{difference of glyphs among CJK}
\begin{tabular}{rl}
Japanese & {\jpnrm 文字,骨練平直。卿響饗嚮,邁進飯餃神祀.才次与}\\
Simplified Chinese &  {\schrm 文字,骨練平直。卿響饗嚮,邁進飯餃神祀.才次与}\\
Traditional Chinese &  {\tchrm 文字,骨練平直。卿響饗嚮,邁進飯餃神祀.才次与}\\
Korean & {\korrm 文字,骨練平直。卿響饗嚮,邁進飯餃神祀.才次}\\
\end{tabular}

\section{switching between CJK and Latin}
\def\bs{{$\backslash$\kern0pt}}

\bs kcatcode=18: 
\kcatcode\ucs"00A7=18% other_kchar
\kcatcode\ucs"0410=18% other_kchar
日本語フォントの§¶
~~
Русский

\bs kcatcode=15: 
\kcatcode\ucs"00A7=15% not cjk character
\kcatcode\ucs"0410=15% not cjk character
Latin §¶
~~
\fontencoding{T2A}\selectfont
Русский

\bs kcatcode=18: 
\kcatcode\ucs"00A7=18% other_kchar
\kcatcode\ucs"0410=18% other_kchar
再び、日本語フォントの§¶
~~
Русский

\bs kcatcode=15: 
\kcatcode\ucs"00A7=15% not cjk character
\kcatcode\ucs"0410=15% not cjk character
Latin §¶
~~
\fontencoding{T2A}\selectfont
Русский

\forcecjktoken
\bs forcecjktoken: 
§¶~~Русский

\disablecjktoken
\bs disablecjktoken: 
§¶~~Русский

\enablecjktoken
\bs enablecjktoken: 
§¶~~Русский

\bs kcatcode=18: 
\kcatcode\ucs"00A7=18% other_kchar
\kcatcode\ucs"0410=18% other_kchar
再び、日本語フォントの§¶
~~
Русский

\forcecjktoken
\bs forcecjktoken: 
§¶~~Русский

\disablecjktoken
\bs disablecjktoken: 
§¶~~Русский

\enablecjktoken
\bs enablecjktoken: 
§¶~~Русский


{ Inside of a group: 
§¶~~Русский
\quad
\disablecjktoken
\bs disablecjktoken: 
§¶~~Русский
}

Outside of the group:
§¶~~Русский


\disablecjktoken
\bs disablecjktoken: 
§¶~~Русский

{ Inside of a group: 
§¶~~Русский
\quad
\enablecjktoken
\bs enablecjktoken: 
§¶~~Русский
}

Outside of the group:
§¶~~Русский

\enablecjktoken
\bs enablecjktoken: 
§¶~~Русский

{ Inside of a group: 
\fontencoding{T2A}\selectfont
§¶~~Русский
\quad
\bs kcatcode=15: 
\kcatcode\ucs"00A7=15% not cjk character
\kcatcode\ucs"0410=15% not cjk character
\fontencoding{T1}\selectfont
§¶
~~
\fontencoding{T2A}\selectfont
Русский
}

Outside of the group:
§¶~~Русский

\bs kcatcode=15: 
\kcatcode\ucs"00A7=15% not cjk character
\kcatcode\ucs"0410=15% not cjk character
\fontencoding{T1}\selectfont
§¶
~~
\fontencoding{T2A}\selectfont
Русский

{ Inside of a group: 
§¶~~Русский
\quad
\bs kcatcode=18: 
\kcatcode\ucs"00A7=18% other_kchar
\kcatcode\ucs"0410=18% other_kchar
%§¶~~%%
%{}§¶~~%%
\relax §¶~~%% この\relax または{}がないとうまく動かない。要調査。
Русский
}

Outside of the group:
§¶~~Русский


\newpage
\section{space between CJK and Latin}
\kcatcode\ucs"00A7=15% not cjk character
\kcatcode\ucs"0410=15% not cjk character

default\\
\fontencoding{T1}\selectfont
今日はStraußのKonzertと
\fontencoding{T2A}\selectfont
ШостаковичのСимфонияを聴いた。

\xkanjiskip1zw plus 1pt minus 1pt
large \bs xkanjiskip\\
\fontencoding{T1}\selectfont
今日はStraußのKonzertと
\fontencoding{T2A}\selectfont
ШостаковичのСимфонияを聴いた。

%\xkanjiskip.25zw plus 1pt minus 1pt
let xspcode 0\\
\xspcode`S=0
\xspcode`K=0
\xspcode`t=0
\xspcode255=0%\ss (ß) in T1 encoding, \cyrya (я) in T2A encoding
\xspcode216=0%\CYRSH (Ш) in T2A encoding
\xspcode247=0%\cyrch (ч)
\xspcode209=0%\CYRS (С)
\fontencoding{T1}\selectfont
今日はStraußのKonzertと
\fontencoding{T2A}\selectfont
ШостаковичのСимфонияを聴いた。


\end{document}
%
% このファイルの大部分は、
% 安田功さんの Utf82TeX のサンプルから引用、
% upTeX用に編集させていただきました。
% upTeXの開発にも大きな刺激になりました。
% ありがとうございました。
% TANAKA, Takuji
%
