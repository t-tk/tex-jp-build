\documentclass{ujarticle}
%\usepackage{lmodern}% Latin Modern
\usepackage[T2A,T1]{fontenc}
\usepackage[russian,japanese]{babel}
\kcatcode`П=15% U+041F:П (Cyrillic)

%%%%%%%%
% set3 関連の制御をコマンドラインから行う
% ①,② のどれかを実行すればよい。
%   ① without set3
%   $ uplatex "\def\withsetthree{no}\input" jbib3-utf8.tex
%   ② with set3
%   $ uplatex jbib3-utf8.tex
%%%%%%
\def\withsetthreetmp{no}

\oddsidemargin0mm
\evensidemargin0mm
\topmargin-15mm
\textwidth162mm
\textheight245mm

\begin{document}
\fontencoding{T1}\selectfont
\section{upTeX用テストデータ}
upbibtexのテスト\cite{森鷗外:百物語}。
upbibtexのテスト\cite{里見弴:極楽とんぼ}。
upbibtexのテスト\cite{国書:丿乀集}。
upbibtexのテスト\cite{グラハム:Unicode™標準入門}。
upbibtexのテスト\cite{test:misc0}。
upbibtexのテスト\cite{test:misc1}。
upbibtexのテスト\cite{test:misc2}。
% set3対応フォント+dviwareなら、「𠮷」もUTF-8で直接書ける。
\ifx\withsetthree\withsetthreetmp\else
upbibtexのテスト\cite{髙島𠮷野}。
\fi

\section{欧文8bit多バイト}
upbibtexのテスト\cite{\detokenize{Lautréamont}}。
upbibtexのテスト\cite{\detokenize{BrüderGrimm}}。
\fontencoding{T2A}\selectfont
upbibtexのテスト\cite{\detokenize{Булгаков}}。
\fontencoding{T1}\selectfont

%\bibliographystyle{jplain}
\bibliographystyle{jalpha}
%\bibliographystyle{jabbrv}
%\bibliographystyle{junsrt}
%\bibliographystyle{jname}

%\bibliographystyle{tipsj}%% 情報処理学会論文誌
%\bibliographystyle{jipsj}%% 情報処理学会欧文論文誌
%\bibliographystyle{tieice}%% 電子情報通信学会論文誌
%\bibliographystyle{jorsj}%% 日本オペレーションズリサーチ学会論文誌

%\fontencoding{T1}\selectfont
\fontencoding{T2A}\selectfont
\bibliography{jbtest}

\end{document}
