% Copyright 2017-2024 Martin Ruckert, Hochschule Muenchen, Lothstrasse 64, 80336 Muenchen
%
% Permission is hereby granted, free of charge, to any person obtaining a copy
% of this software and associated documentation files (the "Software"), to deal
% in the Software without restriction, including without limitation the rights
% to use, copy, modify, merge, publish, distribute, sublicense, and/or sell
% copies of the Software, and to permit persons to whom the Software is
% furnished to do so, subject to the following conditions:
%
% The above copyright notice and this permission notice shall be
% included in all copies or substantial portions of the Software.
%
% THE SOFTWARE IS PROVIDED "AS IS", WITHOUT WARRANTY OF ANY KIND, EXPRESS OR
% IMPLIED, INCLUDING BUT NOT LIMITED TO THE WARRANTIES OF MERCHANTABILITY,
% FITNESS FOR A PARTICULAR PURPOSE AND NONINFRINGEMENT. IN NO EVENT SHALL THE
% COPYRIGHT HOLDERS BE LIABLE FOR ANY CLAIM, DAMAGES OR OTHER LIABILITY,
% WHETHER IN AN ACTION OF CONTRACT, TORT OR OTHERWISE, ARISING FROM, OUT
% OF OR IN CONNECTION WITH THE SOFTWARE OR THE USE OR OTHER DEALINGS IN
% THE SOFTWARE.
%
% Except as contained in this notice, the name of the copyright holders shall
% not be used in advertising or otherwise to promote the sale, use or other
% dealings in this Software without prior written authorization from the
% copyright holders.

\input idxmac.tex
\let\displayverbatimfont=\tt

%                 V E R B A T I M . T E X
%
%          THIS SET OF MACROs IS TAKEN FROM D.E.K.'s TeXBook
%       AND PROVIDES AN EASY WAY OF TYPESETTING TEXTS VERBATIM:
%
% 1. \verbatim<character><text without that character><character>
%    causes the text to be set verbatim using \tt font; 
% 2. if the text uses up all the alphabet, \doubleverbatim macro 
%    can be used instead; this is similar to the previous one but 
%    a pair of characters is now used as a delimiter; 
% 3. in case of emergency \tripleverbatim macro may be of help... 
%
%---------------------------------------------
\def\uncatcodespecials % see D.E.K., pp. 344 and 380
    {\def\do##1{\catcode`##1=12}\dospecials}%
%---------------------------------------------
{\catcode`\^^I=\active \gdef^^I{\ \ \ \ }% TAB character is replaced by
                                         % 4 spaces; it is better than
                                         % nothing, but it does not mimic
                                         % true tabbing satisfactorily---maybe
                                         % some nice day...
 \catcode`\`=\active\gdef`{\relax\lq}}% this line inhibits Spanish 
                                      % ligatures ?` and !` of \tt font
\def\setupverbatim % see D.E.K., p. 381
    {\tt %
     \spaceskip=0pt \xspaceskip=0pt % just in case...
     \catcode`\^^I=\active %
     \catcode`\`=\active %
     \def\par{\leavevmode\endgraf}% this causes that empty lines aren't 
                                  % skipped
     \obeylines \uncatcodespecials \obeyspaces}%
{\obeyspaces \global\let =\ }% this causes that leading blanks aren't 
                             % skipped; cf. also def's of \space, \endgraf,
                             % \lq, \obeyspaces, and \obeylines, 
                             % D.E.K., pp. 351--352
%---------------------------------------------
% see D.E.K., p. 382
\def\doverbatim#1{\def\next##1#1{##1\endgroup}\next}%
\def\verbatim{\begingroup\setupverbatim\doverbatim}%
%----------
\def\dodoubleverbatim#1#2{\def\next##1#1#2{##1\endgroup}\next}%
\def\doubleverbatim{\begingroup\setupverbatim\dodoubleverbatim}%
%----------
\def\dotripleverbatim#1#2#3{\def\next##1#1#2#3{##1\endgroup}\next}%
\def\tripleverbatim{\begingroup\setupverbatim\dotripleverbatim}%
%---------------------------------------------



%%
%% special treatment for @
%%

\def\makeatletter{\catcode`\@=11\relax}
\def\makeatother{\catcode`\@=12\relax}
\makeatletter

%%
%% Conditionals
%%

\newif\ifhint
\newif\ifbook
\newif\ifpdf

% HINT
\expandafter\ifx\csname HINTversion\endcsname\relax
  \hintfalse
\else 
  \hinttrue
\fi

% PDF
\expandafter\ifx\csname pdfliteral\endcsname\relax
  \pdffalse
\else
  \pdftrue
\fi

% Book
\expandafter\ifx\csname book\endcsname\relax
  \ifhint\bookfalse
  \else\ifpdf\bookfalse % This might change
  \else\bookfalse
  \fi\fi
\else
\booktrue
\hintfalse
\pdffalse
\fi

\message{Book: \ifbook true \else false \fi}
\message{PDF: \ifpdf true \else false \fi}
\message{HINT: \ifhint true \else false \fi}

%%
%% Additional fonts 
%% 

\font\tenss=cmss10 % used for the HINT name
\font\tenssbx=cmssbx10 % used for the bold HINT name
\font\largess=cmss12 scaled\magstep1 % used for large HINT name
\font\largessbx=cmssbx10 scaled\magstep2 % used for large bold HINT name
\font\manual=manfnt % font used for the METAFONT logo, etc.
\let\sf\tenss
\def\bf{\let\sf\tenssbx\fam\bffam\tenbf} 


%% Large Font for sections
%\font\largebf=cmb14 scaled\magstep0
%\font\largebf=cmb12 scaled\magstep1
\font\largebf=cmbx12 scaled\magstep1
%\font\largebf=cmbx8 scaled\magstep3
%\font\largeit=cmti14 scaled\magstep0
\font\largeit=cmti12 scaled\magstep1
%\font\largett=cmtt14 scaled\magstep0
\font\largett=cmtt12 scaled\magstep1
%\font\largerm=cmr14 scaled\magstep0
\font\largerm=cmr12 scaled\magstep1
% for mathbolditalic
%\font\tenmib=cmmib10 scaled\magstep0
\font\tenmib=cmmib10 scaled\magstep0
\font\largemib=cmmib10 scaled\magstep1

%% Small Fonts for Figure and Table descriptions
\font\figbf=cmbx9 scaled\magstep0
\font\figit=cmti9 scaled\magstep0
\font\figmit=cmmi9 scaled\magstep0
\font\figsy=cmsy9 scaled\magstep0
\font\figtt=cmtt9 scaled\magstep0
\font\figrm=cmr9 scaled\magstep0
% for mathbolditalic
\font\figmib=cmmib9 scaled\magstep0

\def\largebold{\let\bf\largebf\let\sf\largessbx\bf}

\def\large{\def\rm{\fam0\largerm}\let\bf\largebold\let\it\largeit\let\tt\largett\let\mib\largemib\let\sf\largess%
\textfont0=\largerm\textfont1=\largeit
\setbox\strutbox=\hbox{\vrule height9.5pt depth5.0pt width\z@}\rm}

\def\small{\def\rm{\fam0\figrm}\let\bf\figbf\let\it\figit\let\tt\figtt\let\mib\figmib%
\textfont0=\figrm\textfont1=\figmit\textfont2=\figsy%
\setbox\strutbox=\hbox{\vrule height9.0pt depth4.5pt width\z@}\rm}

\let\tiny\eightrm
\let\mib\tenmib


%%
%% Logos
%%

%\def\Pascal/{{\mc P\kern-.05emascal\spacefactor1000}}
\def\CEE{{\mc C\spacefactor1000}}
\def\Pascal{\leavevmode\hbox{\mc P\kern-.05emascal}}
\def\PDF{\leavevmode\hbox{\mc PDF}}
\def\WEB{\leavevmode\hbox{\tt WEB\spacefactor1000}}
\def\GNU{\leavevmode\hbox{\mc GNU}}
\def\Prote{{\tenrm P\kern-0.1em R\kern-0.15em\raise.11ex\hbox{o}%
  \kern-0.22em T\kern-0.05em E}}
\def\eTeX{$\varepsilon$-\TeX}
\def\HINT{\leavevmode\hbox{\sf HINT\spacefactor1000}}
\def\cweb{\leavevmode\hbox{\tt cweb}}
\def\web2w{\leavevmode\hbox{\tt web2w}}
\def\LaTeX{L\kern-.36em\raise.3ex\hbox{\sc A}\kern-.15em\TeX}%
\def\MF{{\manual META}\-{\manual FONT}}%
\def\TL{\TeX~Live}
\def\kpse{\leavevmode\hbox{\tt kpathsearch\spacefactor1000}}
\def\TUB{{\sl TUGboat\/}}
\def\TUG{\TeX\ \UG}
\def\tug{\acro{TUG}}
\def\UG{Users Group}
\ifpdf
\sanitizecommand{\TeX}{TeX}
\sanitizecommand{\eTeX}{eTeX}
\sanitizecommand{\Prote}{PRoTE}
\fi
%%
%% Special controlsequences
%%

\def\abs#1{\left|#1\right|}
\def\<#1>{$\langle\,$#1$\,\rangle$}
\def\registered{{\ooalign{\hfil\raise.07ex\hbox{\sevenrm R}\hfil\crcr\Orb}}}
\def\hair{\kern.05em\relax} % teeny tiny space
\def\center#1{\bigskip\line{\hfil#1\hfil}\bigbreak}
\chardef\VB=`\| % vertical bar character in a string
\def\^{\ifmmode\mathchar"222 \else\char`^ \fi}%from webmac.tex: pointer or hat
\ifhint
\def\_{{\tt\UL}} % make underline an ordinary character for better searching
\fi

% Bibtex needs this
\def\mbox#1{\leavevmode\hbox{#1}}

% URLs
{\gdef\urldot{.}\catcode`.=13
 \gdef\urlslash{/}\catcode`/=13
 \gdef\urlspecials{\def.{\urldot\penalty 0}\def/{\urlslash\penalty 0}}
}
\def\url{\bgroup\hskip 1cm plus 5cm\penalty -100\hskip -1cm plus -5cm\tt
  \catcode`_=12\catcode`\~=12\catcode`.=13\catcode`/=13\urlspecials}
\def\endurl{\egroup}


% for the index
\def\see#1#2{\hskip 0pt plus 100pt\penalty 0\hskip 0 pt plus -100pt{\it see\/} #1}

% typewriter type for strings
\def\.#1{\leavevmode\hbox{\tt
  \let\\=\BS % backslash in a string
  \let\{=\LB % left brace in a string
  \let\}=\RB % right brace in a string
  \let\~=\TL % tilde in a string
  \let\ =\SP % space in a string
  \let\_=\UL % underline in a string
  \let\&=\AM % ampersand in a string
  \let\^=\CF % circumflex in a string
  \let\|=\VB % vertical bar in a string
  #1\kern.05em}}

%{\rm\gdef\strutdepth{\dp\strutbox}}


%% 
%% Environments
%%

% Quotations (form the standard)
\def\beginquote{\bgroup\narrower\noindent\sl}
\def\endquote{\egroup}

% Enumerate
\newcount\enum
%other styles use \medskip instead of \par
\def\enumerate{\par\bgroup\advance\leftskip by\parindent\enum=0%
   \def\item{\advance\enum by 1\par\enumdest\smallskip\noindent\hbox to 0pt{\hss\the\enum.~}\ignorespaces}}
\def\endenumerate{\medskip\egroup\noindent}

%Itemize
\def\itemize{\par\bgroup\advance\leftskip by\parindent\relax%
   \def\item{\par\smallskip\noindent\llap{$\bullet$\enspace}\ignorespaces}}
\def\enditemize{\medskip\egroup\noindent}

%Float
\long\def\float#1{\midinsert\parindent 0pt\relax#1\endinsert}
%\def\float#1{\midinsert
%\hsize\pagewidth
%\parindent 0pt\vbox{\hbox{\kern-\marginwidth\vbox{#1}}}\endinsert}

%%
%% Redefining cwebmac.tex macros
%%

%defining how to output terminals and nonterminals
\def\ts#1{\ifmmode\hbox{\sc #1}\else{\sc #1}\fi}% terminal symbol
\def\nts#1{{\sl #1\/}}% nonterminal symbol
% redefining the box around verbatim code
\def\vb#1{\hbox{\strut\.{#1}}}
% regular expressions and actions
\newbox\rebox
\def\re#1{\leavevmode\setbox\rebox\hbox{#1\hfill}\ifdim\wd\rebox<7em\wd\rebox=7em\fi\box\rebox\quad}
\def\ac{\global\ind=10\hangindent=\ind em\relax}
\def\eac{\global\ind=3\hangindent=\ind em\relax}
% yskip is usually a smallskip = 3pt pus 1pt minus 1pt
% I give it some extra stretchability
%\def\yskip{\vskip 3pt plus 3pt minus 1pt}
\let\yskip\smallbreak

% redefine \5 for an optional break without the 2.5em reduced indentation
%\def\5{\hskip 0pt plus 2.5em\penalty-1\hskip 0pt plus -2.5em\kern0.5em\ignorespaces}% optional break
%redefine \5 to give some shrinkability
% from hitex and hint style:
%\def\5{\hfil\penalty-1\hskip 0pt minus 6pt\hfilneg\kern2.5em\copy\bakk\ignorespaces}% optional break
\def\5{\hfil\penalty10\hfilneg\hskip 0pt minus 6pt\kern2.5em\copy\bakk\ignorespaces}% optional break
\def\J{}% dont show @& in the tex output


\def\Y{\par\setsecstartref\yskip}

% start sections
\def\stsec{\rightskip=0pt % get out of C mode (cf. \B)
  \sfcode`;=1500 \pretolerance 200 \hyphenpenalty 50 \exhyphenpenalty 50 %
}

% the box for the points to operator "->" in C 
\setbox\MGbox=\hbox{$\rightarrow$}

% how to display NULL
\def\NULL{\tt NULL}

%other styles make this a \par
\def\note#1#2.{}
%\Y\noindent{\hfill%
%    \baselineskip10pt\tiny#1~\ifacro{\pdfnote#2.}\else#2\fi.\par}}

% put the section number to the right
\newif\ifshowsecstartref \showsecstartreffalse
\newif\ifsetsecstartref \setsecstartreffalse

\def\secstartref{% used in \MN
  \ifshowsecstartref
  \vadjust{\vtop to 0pt{\vss\hbox to \hsize{\hfill\tiny(\secstar)}\kern 0pt}}
  \global\showsecstartreffalse
  \fi
}

\def\setsecstartref{% used in \Y
  \ifsetsecstartref
  \codedest
  \global\setsecstartreffalse
  \fi
}


%other styles make this minus 10pt
\def\B{\rightskip=0pt plus 100pt minus 0pt % go into C mode
  \sfcode`;=3000
  \pretolerance 10000
  \hyphenpenalty 1000 % so strings can be broken (discretionary \ is inserted)
  \exhyphenpenalty 1000
  \global\ind=2 \1\ \unskip
}

% formating of C comments
\def\C#1{\5\hfill$/\ast\,${\cmntfont #1}$\,\ast/$}

% section begin
\def\M#1{\MN{#1}\ifon\stsec\smallskip
%\everypar{{\setbox0=\lastbox}\everypar{}}% no indentation in the next paragraph
\noindent\ignorespaces}% beginning of section

\def\N#1#2#3.{\MN{#1}\ifon\stsec\smallskip
%\everypar{{\setbox0=\lastbox}\everypar{}}% no indentation in the next paragraph
\noindent{\bf#3\quad}\ignorespaces}% beginning of section

\def\MN#1{\smallskip
 {\xdef\secstar{#1}\let\*=\empty\xdef\secno{#1}}%
  \gdef\thecode{#1}% common code for \M, \N
  \ontrue\showsecstartreftrue\setsecstartreftrue}

% Used in section text
\def\U{\rightnote{Used in}} % xref for use of a section
\def\Us{\rightnote{Used in}} % xref for uses of a section
% replacing \note in \U and \Us by \rightnote 
%\def\rightnote#1#2.{\vskip-\baselineskip\vtop to 0pt{\vss\hbox to \hsize{\hfill
%    \tiny#1~\ifacro{\pdfnote#2.}\else#2\fi.}\kern 0pt}}
%\def\rightnote#1#2.{\vskip-\baselineskip\hbox to \hsize{\hfill
%    \tiny#1~\ifacro{\pdfnote#2.}\else#2\fi.}}
%\def\rightnote#1#2.{\penalty1000\discretionary{}{\hbox{}}{\kern 2em}\penalty1000\hfill
%    \hbox{\tiny#1~\ifacro{\pdfnote#2.}\else#2\fi.}}
\def\rightnote#1#2.{%
     \penalty1000\discretionary{}{\hbox{}}{\kern 4em}\penalty1000\hfill
     \hskip -4em plus 4em\hbox{\tiny #1~\codenote#2.}.}

\def\X#1:#2\X{\setsecstartref
  \ifmmode\gdef\XX{\null$\null}\else\gdef\XX{}\fi %$% section name
  \XX$\langle\,${\let\I=\ne#2\kern.5em\coderef{#1}$\rangle$\XX}%
  \secstartref}


% how to display hex numbers
\def\hex{\hbox{$^{\scriptstyle\#}$\tt\aftergroup}} % CWEB style

\def\(#1){} % this is used to make section names sort themselves better


%%
%% Dimensions to achive the desired layout
%%


% the other variables are defined in cwebmac.tex
\newdimen\headheight 
\newdimen\footheight 
\newdimen\leftmargin
\newdimen\rightmargin
\newdimen\topmargin
\newdimen\bottommargin
\newdimen\marginwidth

\parskip 0pt plus .8pt

\leftmargin=22mm
\rightmargin=22mm
\topmargin=15mm
\bottommargin=27mm
\marginwidth=0pt % we do not have notes in the margin

\headheight=12pt % Text in Head
\advance\headheight+13pt % Abstand 1pt, Linie 0.47pt und Abstand zum Text 11.53pt
\footheight=0pt % Text in Footer

\pagewidth=6.69in
\advance\pagewidth-\leftmargin % left margin/offset
\advance\pagewidth-\rightmargin % left margin/offset

% 6.69in = 169.926
% - 2*22mm = 125.926mm
%
% 9.61in = 244.094
% - 15mm  -27 mm = 202.094
% (12+13)pt = 8.786 mm
% -8.786 = 193.308 mm

\fullpageheight=9.61in
\advance\fullpageheight-\topmargin% topmargin
\advance\fullpageheight-\bottommargin% bottommargin

\pageheight=\fullpageheight
\advance\pageheight-\headheight
\advance\pageheight-\footheight

\def\setpage{\hsize\pagewidth\advance\hsize-\marginwidth\vsize\pageheight} % use after changing page size

\setpage

% setting the papersize for postscript and pdf

\ifbook
  %\overfullrule=0pt
  \pdfpageheight=9.61in
  \pdfpagewidth=6.69in
  \pdfhorigin=\leftmargin
  \pdfvorigin=\topmargin
  \pdfcompresslevel=9
  \pdfdecimaldigits=4
  \pdfpkresolution=1200
  \pdfimageresolution=1200
\else\ifpdf
  \pdfpageheight=9.61in
  \pdfpagewidth=6.69in
  \pdfhorigin=\leftmargin
  \pdfvorigin=\topmargin
  \pdfcompresslevel=9
  \pdfdecimaldigits=4
  \pdfpkresolution=1200
  \pdfimageresolution=1200
\else
   \special{papersize=6.69in,9.61in}%  postscript und ps2pdf
\fi\fi



%%
%% Page layout
%%

\long\def\leftmark#1#2{#1}
\long\def\subsectionmark#1#2{#2}
\long\def\rightmark#1#2{\if0#1\else\expandafter\subsectionmark\botmark\fi}

\def\lheader{\mainfont\strut
\thepageno\hfill\expandafter\leftmark\firstmark} % top line on left-hand pages

\def\rheader{\mainfont\strut
\expandafter\rightmark\firstmark\hfill\thepageno} % top line on right-hand pages

\let\page=\pagebody 
%\raggedbottom
\normalbottom
%\def\page{\box255 }\normalbottom % faster, but loses plain TeX footnotes
\def\normaloutput#1#2#3{\ifodd\pageno\hoffset=\pageshift\fi
 \shipout\vbox{
   \vbox to\fullpageheight{\pagelabel
     \iftitle\global\titlefalse
     \else\hbox{\vbox to \headheight{
       \hbox to \pagewidth{\ifodd\pageno #3\else#2\fi}
       \vskip 1pt\relax
       \nointerlineskip
       \hrule height .47pt
%      \hbox{\hrule}%fill\psfig{file=image/topline.eps}\hfil}%
       \vfil}}%
     \fi
     \nointerlineskip
     \hbox{\kern\marginwidth\vbox to \pageheight{#1}}}} % parameter #1 is the page itself
  \global\advance\pageno by1}

\def\nomarginoutput#1#2#3{%
 \ifodd\pageno\hoffset=\pageshift\fi
 \shipout\vbox to\fullpageheight{\pagelabel\nointerlineskip
  \iftitle\global\titlefalse
    \hbox{\vbox to \headheight{\vfil}}
  \else
    \hbox{\vbox to \headheight{\nointerlineskip
      \hbox to \pagewidth{\ifodd\pageno#3\else#2\fi}%
      \vskip 1 pt
      \nointerlineskip
      \hrule height .47pt
%    \hbox{\psfig{silent=,bbllx=0pt,bblly=0pt,bburx=368pt,bbury=2pt,file=image/topline.eps}\hfil}%
    \vfil}}%
  \fi\nointerlineskip #1 % parameter #1 is the page itself
%  \vss
%    \hbox to \pagewidth{\ifodd\pageno\else#2\fi}%
%    \nointerlineskip
%    \hbox{\psfig{file=image/topline.eps}\hfil}%
}% 
\global\advance\pageno by1}

%% switch to global nomargin
\let\normaloutput\nomarginoutput
\def\sectionbox#1{\hbox{#1}}
\marginwidth=0mm
\setpage


%%
%% Frontmatter and mainmatter
%%
\pageno=1
\def\thepageno{\romannumeral\pageno}
\def\frontmatter{\pageno=4\def\thepageno{\romannumeral\pageno}}
\def\mainmatter{
\mark{{\sectionname}{\subsectionname}}
%\message{Mark {\sectionname}{\subsectionname}}
\vfil\break\ifodd\pageno\pageno=1\else\pageno=0\fi
\def\thepageno{\the\pageno}%
\sectioncount=0
\mark{{0}{0}}%
%\hbox{Mark empty empty}%
\gdef\sectionname{}
\gdef\subsectionname{}
\gdef\codetitle{}
\gdef\subcodetitle{}
}




%%
%% Sections
%%

\newcount\sectioncount
\sectioncount=0
\newcount\subsectioncount
\subsectioncount=0
\newcount\subsubsectioncount
\subsubsectioncount=0

\newskip\abovesecskip 
\newskip\belowsecskip 
\newskip\abovesubsecskip
\newskip\belowsubsecskip 

%\abovesecskip= 6ex plus 1ex minus .2ex %space above the section
%\belowsecskip=3.7ex plus .2ex% space after section
\abovesecskip= 0.15\vsize plus 5ex minus 1ex %space above the section
\belowsecskip=10pt plus 2pt% space after section
\abovesubsecskip=9pt plus 4pt minus 2pt%space above the section
\belowsubsecskip=3pt plus 1pt minus 0.5pt% space after section


\newif\ifappendix
\appendixfalse

\mark{{0}{0}}\vskip 0pt minus 100pt
%\hbox{Mark empty empty}%
\def\thesection{\the\sectioncount}
\def\topsection{\the\sectioncount}

%\def\sectionbox#1{\hskip-\marginwidth\hbox to \marginwidth{#1\hfil}}

\def\heading#1#2{%
\def\secno{{\noindent\large\bf\strut{#1}#2}}%
\par\ifhmode\unskip\fi%     end paragraph and remove vertical space
\penalty-500
\vskip 0pt plus 72pt%allow some empty space at the bottom
\penalty-500
\vskip 0pt plus -72pt%    room for stetching and a page break
\hbox{}\vskip\abovesecskip%   space above the section
%%\titletrue % omits page header for section
%\hbox{Mark null null}%
\mark{{0}{0}}%
\secno% The section title
\mark{{\sectionname}{\sectionname}}%
%\message{Mark heading {\sectionname}{\subsectionname}}
\nobreak\vskip\belowsecskip% space after section
\everypar{{\setbox0=\lastbox}\everypar{}}% no indentation in the next paragraph
\ignorespaces}



\newcount\plaincount
\def\plainsection#1{%
\mark{{\sectionname}{\subsectionname}}%
%\message{Mark plain {\sectionname}{\subsectionname}}
\subsubsectioncount=0\subsectioncount=0%
\advance\plaincount by 1\relax
\def\thesection{}
\sectionpage%
\gdef\sectionname{#1}%
\gdef\subsectionname{#1}%
\gdef\codetitle{#1}%
\gdef\subcodetitle{#1}%
\tocsection{0}{}{#1}
\heading{}{#1}%
}


\gdef\sectionname{}
\gdef\subsectionname{}
\gdef\codetitle{}%
\gdef\subcodetitle{}

\def\sectionpage{%
\vfil\break
\ifodd\pageno\else
\hbox{}%
\titletrue
\fi
\vfil\break
}


\def\section#1{%
\mark{{\sectionname}{\subsectionname}}%
%\message{Mark section {\sectionname}{\subsectionname}}
\subsubsectioncount=0\subsectioncount=0%
\advance\sectioncount by 1%updating counts
\let\thesection\topsection
\sectionpage%
\gdef\sectionname{\topsection\quad#1}%
\gdef\subsectionname{\thesection\quad#1}%
\gdef\codetitle{#1}%
\gdef\subcodetitle{#1}%
\ifnum\sectioncount=1\ifappendix\tocsection{0}{}{Appendix}\heading{}{Appendix}\fi\fi
\tocsection{0}{\thesection}{#1}%
\heading{\thesection~}{#1}%
}


\def\subsection#1{%
\subsubsectioncount=0\advance\subsectioncount by 1%updating counts
\def\thesection{\topsection.\the\subsectioncount}  
\def\secno{{\noindent\bf\strut\sectionbox{\thesection~}#1}}%
\par\ifhmode\unskip\fi%     end paragraph and remove vertical space
\tocsection{1}{\thesection}{#1}%
\penalty-200
\vskip 0pt plus 36pt% allow some empty space at the bottom
\penalty-200
\vskip 0pt plus -36pt%    room for stetching and a page break
\gdef\subsectionname{\thesection\quad#1}%
\mark{{\sectionname}{\subsectionname}}%
%\message{Mark subsection {\sectionname}{\subsectionname}}
\vskip\abovesubsecskip%   space above the subsection
\secno% The subsection title
\gdef\subcodetitle{#1}%
%\mark{{\sectionname}{\subsectionname}}%
%\hbox{Mark {\sectionname}{\subsectionname}}%
\nobreak\vskip\belowsubsecskip% space after subsection
\everypar{{\setbox0=\lastbox}\everypar{}}% no indentation in the next paragraph
\ignorespaces}


\def\subsubsection#1{%
%\mark{{\sectionname}{\subsectionname}}%
%\hbox{Mark {\sectionname}{\subsectionname}}%
  \advance\subsubsectioncount by 1%updating counts
\def\thesection{\topsection.\the\subsectioncount.\the\subsubsectioncount}  
\def\secno{{\noindent\it\strut\sectionbox{\thesection~}#1}}%
\par\ifhmode\unskip\fi%     end paragraph and remove vertical space
\tocsection{2}{\thesection}{#1}%
\penalty-200
\vskip 0pt plus 36pt% allow some empty space at the bottom
\penalty-200
\vskip 0pt plus -36pt%    room for stetching and a page break
\vskip\abovesubsecskip%   space above the subsection
\secno% The subsubsection title
\nobreak\vskip\belowsubsecskip% space after subsection
%\everypar{{\setbox0=\lastbox}\everypar{}}% no indentation in the next paragraph
%\noindent
\ignorespaces}

%%
%% Appendix
%%

\def\Alphanum#1{%
  \ifcase#1\or A\or B\or C\or D\or E\or F\or G\or H\or I\or J\or
   K\or L\or M\or N\or O\or P\or Q\or R\or S\or T\or U\or V\or W\or X\or
    Y\or Z\else\number#1\fi}

\def\appendix{%
\mark{{\sectionname}{\subsectionname}}%
%\message{Mark appendix {\sectionname}{\subsectionname}}
\sectioncount=0
\subsectioncount=0
\gdef\thesection{\Alphanum\sectioncount}%
\gdef\topsection{\Alphanum\sectioncount}%
\appendixtrue
}

%%
%% Crossreference of Identifiers and Sections
%%
\iftrue %% No Crossreference of identifiers
  \def\inx{\def\lr{L}} % this tells whether the left or right column is next
\else
\def\inx{%
  \hsize=\pagewidth
  \def\page{\box255 } \normalbottom
  \output{\ifpagesaved\normaloutput{\box\sbox}\lheader\rheader\fi
    \global\setbox\sbox=\page \global\pagesavedtrue}
  \pagesavedfalse 
  \plainsection{Crossreference of Identifiers}% we are beginning the index
  \mark{{0}{0}}%
%\hbox{Mark empty empty}%
  \eject % eject the page-so-far and predecessors
  \setbox\sbox\vbox{\unvbox\sbox} % take it out of its box
  \vsize=\pageheight \advance\vsize by -\ht\sbox % the remaining height
  \hsize=.5\pagewidth \advance\hsize by -10pt
    % column width for the index (20pt between cols)
  \parfillskip 0pt plus .6\hsize % try to avoid almost empty lines
  \def\lr{L} % this tells whether the left or right column is next
  \output{\if L\lr\global\setbox\lbox=\page \gdef\lr{R}
    \else\nomarginoutput{\vbox to\pageheight{\box\sbox\vss
        \hbox to\pagewidth{\box\lbox\hfil\page}}}\lheader\rheader
    \global\vsize\pageheight\gdef\lr{L}\global\pagesavedfalse
    \mark{{Crossreference of Identifiers}{Crossreference of Identifiers}}%
    \fi}
  \parfillskip 0pt plus 1fil
  \let\topsecno=\nullsec
  \message{Crossreference of Identifiers:}
  \parskip 0pt plus .5pt
  \outer\def\I##1, ##2.{\par\hangindent2em\noindent##1:\kern1em
    \ifbook##2\else\ifpdf\pdfnote##2.\else##2\fi\fi.} % index entry
  \def\[##1]{$\underline{##1}$} % underlined index item
  \rm \rightskip0pt plus 2.5em \tolerance 10000 \let\*=\lapstar
  \hyphenpenalty 10000 \parindent0pt
  \small
  \readindex
}
\fi

\def\fin{%
  \if L\lr\mark{{}{}}\fi
  \vfill\eject % complete the current column.
   \mark{{}{}}
  \if R\lr\null\vfill\eject\fi % if necessarry add a right column
  \setpage
  \def\page{\box255 } \normalbottom
  \output={\nomarginoutput\page\lheader\rheader}
}
\ifbook
\let\crosssections=\relax
\else
\def\crosssections{
  \plainsection{Crossreference of Code}% this is done when we are ending the index
  \parindent 0pt
  \parfillskip 0pt plus 1fil
  \let\topsecno=\nullsec
  \message{Crossreference of Code:}
  \def\note##1##2.{\hfil\penalty-1\hfilneg\quad{\tiny##1~\ifbook##2\else
      \ifpdf{\pdfnote##2.}\else
            \ifhint\codenote##2.\else##2.\fi\fi\fi}}
  \def\Q{\note{Cited in section}} % crossref for mention of a section
  \def\Qs{\note{Cited in sections}} % crossref for mentions of a section
  \def\U{\note{Used in}} % crossref for use of a section
  \def\Us{\note{Used in}} % crossref for uses of a section
  \def\I{\par\hangindent 2em}\let\*=*
  \def\X##1:##2\X{\ifmmode\gdef\XX{\null$\null}\else\gdef\XX{}\fi %$% section name
  \XX$\langle\,${\let\I=\ne##2}$\,\rangle$\XX\quad{\tiny Defined in~\codenote##1.}}
  \readsections
}
\fi



%%
%% Table of Content
%%

% we separate the identification of a section used in links
%from the identification on paper, because plain sections dont
%have the latter.
\newcount\toccount
\toccount=0

\def\tocsection#1#2#3{% depth sectionnumber sectiontitle
\global\advance\toccount by 1%
\newdest{SC.\the\toccount}%
\ifhint
  \immediate\write\cont{\noexpand\ZZ {#1}{#2}{#3}% write to contents file
    {\noexpand\thepageno}{\the\toccount}}
\else
\edef\next{\write\cont{\noexpand\ZZ {#1}{#2}{#3}% write to contents file
    {\noexpand\thepageno}{\the\toccount}}}\next % \tocline{depth}{sec}{title}{page}{toccount}
\fi
}

\output{
  \ifhint
  \shipout\box255\relax% otherwise I loose open, write, or close commands
  \else
  \setbox0=\page % the first page is garbage
  \fi
%  \openout\cont=\contentsfile
%  \write\cont{\catcode `\noexpand\@=11\relax}   % \makeatletter
  \global\output{\normaloutput\page\lheader\rheader}}


\newbox\tocbox
\def\maketoc{%
  \typeout{Reading table of contents}
  \setbox\tocbox\vbox{%
  \let\ZZ=\tocline
  \ifbook\else\ifpdf\startpdf\fi\fi
  \readcontents\relax
  }
  \typeout{Writing \contentsfile}
  \immediate\openout\cont=\contentsfile\relax
  \immediate\write\cont{\catcode `\noexpand\@=11\relax} 
}

\def\tableofcontent{\plainsection{Contents}\ifhint\HINTdest name {HINT.home}\fi
\unvbox\tocbox  
}


\def\tocline#1#2#3#4#5{% depth sectionnumber sectiontitle page toccount
\ifnum#1=0 \smallskip\vskip 0pt plus 12pt\penalty -100\vskip 0pt plus -12pt\fi
\noindent
\line{%
\ifnum#1=0%
  \hbox to 2em{\bf#2\hfill}{\bf#3~}\hfill
     \hbox to 2em{\hss\bf\tocref{#1}{#3}{#4}{#5}}%
\else \ifnum#1=1%
   \kern 2em\hbox to 3em{\rm#2\hfill}{\rm#3~}\leaders\hbox to .5em{.\hfil}\hfill
      \hbox to 2em{\hss\tocref{#1}{#3}{#4}{#5}}%
\else
   \kern 2em\hbox to 5em{\rm#2\hfill}{\it#3~}\leaders\hbox to .5em{.\hfil}\hfill
      \hbox to 2em{\hss\tocref{#1}{#3}{#4}{#5}}\fi\fi
}}



%%
%% Figures and Tables
%%

\def\fig#1{\global\advance\figcount by 1%
\ifvmode\noindent\fi
\def\captype{Fig}%
\capskip=1em%
\vbox{\nointerlineskip\figdest
\hbox{\noindent#1\hskip -2pt\hskip 0pt plus 2pt minus 2pt}%
\medskip
\rlap{\small\unhbox\capbox}%
}}

\def\tab#1{%
\ifvmode\noindent\fi
\def\captype{Tab}%
\capskip=0pt%
\vbox{#1\medskip\rlap{\small\unhbox\capbox}}}

\def\table#1#2{\global\advance\tabcount by 1%
\ifvmode\noindent\fi
\def\captype{Tab}%
\capskip=0pt%
\vbox{\offinterlineskip\tabdest
   \halign{\large\strut
   \vrule##&&\quad\hfil##\hfil\quad\vrule\cr
   \noalign{\hrule}#2\noalign{\hrule}}%
   \medskip
   \tabcaption{#1}%
  \rlap{\small\unhbox\capbox}%
  }%
}

\def\captype{}
\newdimen\capskip
\capskip=1em
\newbox\capbox
\setbox\capbox\hbox{}
\newcount\figcount
\figcount=0
\newcount\tabcount
\tabcount=0

\def\nextfigcaption#1{\small\strut\it\hskip\capskip\figindex{\captype}{\the\figcount}{#1}\captype.~\the\figcount:}

\def\caption#1{\global\setbox\capbox\hbox{\nextfigcaption{#1} #1}}

\def\nexttabcaption#1{\small\strut\it\hskip\capskip\tabindex{\captype}{\the\tabcount}{#1}\captype.~\the\tabcount:}
\def\tabcaption#1{\global\setbox\capbox\hbox{\nexttabcaption{#1} #1}}


%\long\def\leftfig#1#2#3{%
%\ifvmode\noindent\fi
%\def\captype{Fig}%
%\capskip=1em%
%\vbox{%
% \tabskip=0pt\halign to \hsize{%
% \vtop{\vskip 0pt\parskip=0pt\hsize=62.5mm\relax
%  ##}\tabskip=5mm plus 2pt minus 2pt&
% \vtop{\vskip 0pt\parskip=0pt\hsize=62.5mm\relax##}\tabskip=0pt\cr
% #1&\capskip=0pt\nextfigcaption{#2} #3\cr}\medbreak}}

\long\def\leftright#1#2#3#4{%
\ifvmode\noindent\else\par\noindent\fi
\vbox{\nointerlineskip
  \advance\hsize by -\leftskip\advance\hsize by -\rightskip%
  \leftskip=0pt\rightskip=0pt%
  \hbox to \hsize{\strut
   \raise\ht\strutbox
   \vtop{\vskip 0pt\parskip=0pt\hsize=#1\hsize\advance\hsize by -2.5mm\relax
   \noindent\strut\ignorespaces#3\vfill}\hfil
   \raise\ht\strutbox
   \vtop{\vskip 0pt\parskip=0pt\hsize=#2\hsize\advance\hsize by -2.5mm\relax
   \noindent\strut\ignorespaces#4\vfill}}}}


\long\def\leftfig#1#2#3{%
\def\captype{Fig}%
\capskip=0pt%
\leftright{0.5}{0.5}{#1\kern -2pt\hskip 2pt plus 2pt minus 2pt}{\nextfigcaption{#2}\strut #3}}


\def\subindex#1{%
  \par\ifhmode\unskip\fi%     end paragraph and remove vertical space
  \vskip\abovesubsecskip%   space above the subsection
  \vskip 0pt plus 72pt% allow some empty space at the bottom
  \penalty-200\vskip 0pt plus -72pt %    room for stetching and a page break
  \noindent{\bf\strut#1}%
  \nobreak\vskip\belowsubsecskip% space after subsection
  \everypar{{\setbox0=\lastbox}\everypar{}}% no indentation in the next paragraph
  \ignorespaces
}

\newwrite\@figfile
\newwrite\@tabfile

\def\makefigindex{%
  \immediate\openout\@figfile=\jobname.fig\relax
  \def\figindex{\@bsphack\begingroup\@sanitize\@wrfigindex\@figfile}%
  \immediate\openout\@tabfile=\jobname.tab\relax
  \def\tabindex{\@bsphack\begingroup\@sanitize\@wrfigindex\@tabfile}%
  \typeout{Writing index files \jobname.fig \jobname.tab}
}

\newdimen\iboxsize
\def\thefigindex{%

  \immediate\closeout\@figfile\relax
  \immediate\closeout\@tabfile\relax
  %\small\baselineskip=11pt
  \plainsection{List of Figures and Tables}
  \makeatletter
  \iboxsize=3.7em\relax
  \subindex{Figures}
  \def\indexentry##1##2##3##4{%  captype, fig/tab count, caption, page
    \noindent
    \line{\hbox to \iboxsize{%
        \rm##1.~##2:\hfil}{\rm ##3}\ \leaders\hbox to .5em{.\hfil}%
      \hfill\hbox to 1.4em{\hss
        \ifbook##4\else
        \ifpdf
        \pagelnk{##4}%
        \else\ifhint\figlink{##2}
        \else##4\fi\fi\fi
       }}\par
  }%
  \input\jobname.fig\relax
  \iboxsize=3.9em\relax
  \subindex{Tables}
  \def\indexentry##1##2##3##4{%  captype, fig/tab count, caption, page
    \noindent
    \line{\hbox to \iboxsize{%
        \rm##1.~##2:\hfil}{\rm ##3}\ \leaders\hbox to .5em{.\hfil}%
      \hfill\hbox to 1.4em{\hss
        \ifbook##4\else
        \ifpdf
        \pagelnk{##4}%
        \else\ifhint\tablink{##2}
        \else##4\fi\fi\fi
       }}\par
  }%
  \input\jobname.tab\relax
  \makefigindex
}


\def\@wrfigindex#1#2#3#4{% file, captype, fig/tab count, caption
   \xdef\gtempa{\write#1{\string\indexentry{#2}{#3}{#4}{\noexpand\thepageno}}}\endgroup\gtempa
   \if@nobreak\ifvmode\@nobreak\fi\fi\@esphack}

%%
%% Format Definitions
%%

%\newbox\codebox
\newwrite\@getfile
\newwrite\@putfile
\newwrite\@redfile
\newwrite\@wrtfile

\def\makecode{%
  \immediate\openout\@getfile=\jobname.get\relax
  \immediate\openout\@putfile=\jobname.put\relax
  \immediate\openout\@redfile=\jobname.red\relax
  \immediate\openout\@wrtfile=\jobname.wrt\relax
  \typeout{Writing index files \jobname.get \jobname.put \jobname.red \jobname.wrt}
}

\def\thecodeindex{%
  \immediate\closeout\@getfile\relax
  \immediate\closeout\@putfile\relax
  \immediate\closeout\@redfile\relax
  \immediate\closeout\@wrtfile\relax
  \def\indexentry##1##2##3##4{% codecount, depth, pagenumber, section
    \ifnum##2=0 \smallskip\vskip 0pt plus 12pt\penalty -100\vskip 0pt plus -12pt\fi
    \noindent
    \line{%
       \ifnum##2=0##4\hfill
       \else\qquad\rm##4\ \leaders\hbox to .5em{.\hfil}\hfill
       \hbox to 1.4em{\hss\pagelnk{##3}}
       \fi
    }\par
   }%
   \section{Format Definitions}
  \makeatletter
  \subsection{Reading the Long Format}\label{codeindex}
  \input\jobname.red.srt\relax
  \subsection{Writing the Long Format}
  \input\jobname.wrt.srt\relax
  \subsection{Reading the Short Format}
  \input\jobname.get.srt\relax
  \subsection{Writing the Short Format}
  \input\jobname.put.srt\relax
}


\newcount\codecount
\codecount=0

\ifhint
\def\@wrcodeindex#1#2#3#4{% file, depth, pagenumber, [sub]sectionname
  \global\advance\indexcount by 1\indexdest
  \xdef\gtempa{\write#1{\string\indexentry{#3}{#2}{\the\indexcount}{#4}}}\endgroup\gtempa\@esphack
}
\else
\def\@wrcodeindex#1#2#3#4{% file, depth, pagenumber, [sub]sectionname
  \xdef\gtempa{\write#1{\string\indexentry{#3}{#2}{\noexpand\thepageno}{#4}}}\endgroup\gtempa\@esphack
}
\fi

\def\getindex{\@bsphack\begingroup\@sanitize\@wrcodeindex\@getfile}%
\def\putindex{\@bsphack\begingroup\@sanitize\@wrcodeindex\@putfile}%
\def\redindex{\@bsphack\begingroup\@sanitize\@wrcodeindex\@redfile}%
\def\wrtindex{\@bsphack\begingroup\@sanitize\@wrcodeindex\@wrtfile}%


\def\codelabel#1#2#3{
  \getindex{#1}{#2}{#3}%
  \putindex{#1}{#2}{#3}%
  \redindex{#1}{#2}{#3}%
  \wrtindex{#1}{#2}{#3}}
\def\getsymbol{$\cdots\Longrightarrow$}
\def\putsymbol{$\Longrightarrow\cdots$}
\def\redsymbol{$-{}-{}-{}\Longrightarrow$}
\def\wrtsymbol{$\Longrightarrow{}-{}-{}-$}
\def\getcode{\codesection{\getsymbol}{Reading the short format}\getindex{1}{\the\codecount}{\subcodetitle}}
\def\putcode{\codesection{\putsymbol}{Writing the short format}\putindex{1}{\the\codecount}{\subcodetitle}}
\def\writecode{\codesection{\wrtsymbol}{Writing the long format}\wrtindex{1}{\the\codecount}{\subcodetitle}}
\def\readcode{\codesection{\redsymbol}{Reading the long format}\redindex{1}{\the\codecount}{\subcodetitle}}
\newdimen\coderule
\def\codesection#1#2{%
  \par\ifhmode\unskip\fi%     end paragraph and remove vertical space
  \penalty-200
  \vskip 0pt plus 72pt% allow some empty space at the bottom
  \penalty-200
  \vskip 0pt plus -72pt%    room for stetching and a page break
\vskip\abovesubsecskip%   space above the subsection
%\line{\setbox0=\hbox{#1}%
%\coderule=\hsize\advance\coderule by -\wd0\advance\coderule by -2em%
%$\vcenter{\hrule width \coderule}$\hss\box0}
\line{{\it #2\/}:\hfill#1}
\nobreak\vskip\belowsubsecskip\nobreak% space after subsection
%\everypar{{\setbox0=\lastbox}\everypar{}}% no indentation in the next paragraph
%\noindent
\ignorespaces}

\def\hascode{\global\advance\codecount by 1%
\codelabel{0}{\the\codecount}{\codetitle}}%


%%
%% Makeindex and the Index
%%

\newwrite\@indexfile

% new jobname for index file
\def\makeindex{
  \immediate\openout\@indexfile=\jobname.inx%
  \def\index{\@bsphack\begingroup\@sanitize\@wrindex\@indexfile}%
  \typeout{Writing index file \jobname.inx }%
}

\def\@idxitem{\par\hangindent 40pt}
 
\newif\ifnextindex
\nextindexfalse

\def\beginindex{%
\ifhint\relax\else
  \def\page{\box255} \normalbottom
  \output{\ifpagesaved\normaloutput{\box\sbox}\lheader\rheader\fi
    \global\setbox\sbox=\page \global\pagesavedtrue}
  \pagesavedfalse \eject % eject the page-so-far and predecessors
  \setbox\sbox\vbox{\unvbox\sbox} % take it out of its box
  \vsize=\pageheight \advance\vsize by -\ht\sbox % the remaining height
  \hsize=.5\pagewidth \advance\hsize by -10pt
    % column width for the index (20pt between cols)
  \parfillskip 0pt plus .6\hsize % try to avoid almost empty lines
  \def\lr{L} % this tells whether the left or right column is next
  \output{\if L\lr\global\setbox\lbox=\page \gdef\lr{R}
    \else\nomarginoutput{\vbox to\pageheight{\box\sbox\vss
        \hbox to\pagewidth{\box\lbox\hfil\page}}}\lheader\rheader
    \global\nextindextrue
    \global\vsize\pageheight\gdef\lr{L}\global\pagesavedfalse\fi}
\fi
\message{Index:}
\mark{{0}{0}}
\noindent\par
\begingroup
  \rightskip 0pt plus 12pt
  \parskip 0pt plus .5pt
  \hyphenpenalty 10000 \parindent0pt
  \small\baselineskip=11pt
\def\item{\par\hangindent 40pt\relax\ifnextindex\mark{{Index}{Index}}\fi}%
\def\subitem{\par\hangindent 40pt\hskip 20pt\relax}%
\def\subsubitem{\par\hangindent 40pt\hskip 30pt\relax}%
\def\indexspace{\par\vskip 12pt plus 6pt minus 4pt\vskip 12pt\goodbreak\vskip -12pt\relax}%
\def\indexheading##1{{\tenbf ##1}\nobreak\vskip 3pt\relax}
}
\def\endindex{%
\endgroup
  \vfill\eject % complete the current column.
\ifhint\relax\else
  \if R\lr\null\vfill\eject\fi % if necessarry add a right column
\fi
}


%%
%% Label page, section, toccount, figure, table, code, enum
%%
\newcount\cdcount
\cdcount=0

\newdimen\labelskip
\def\label#1{% writes \labeldef{name}{{page}{section}{toccount}{figure}{table}{code}{enum}} to aux
  \ifhint
  \global\advance\indexcount by 1\indexdest
     \begingroup\@readauxfile
   \edef\next{\@writeaux{\string\labeldef{#1}{{\the\indexcount}%
         {\thesection}{\the\toccount}{\the\figcount}{\the\tabcount}{\the\cdcount}{\the\enum}}}\endgroup}%
  \else
   \begingroup\@readauxfile
   \edef\next{\@writeaux{\string\labeldef{#1}{{\noexpand\thepageno}%
         {\thesection}{\the\toccount}{\the\figcount}{\the\tabcount}{\the\cdcount}{\the\enum}}}\endgroup}%
  \fi
   \ifvmode %make \removelastskip work after \label 
     \labelskip=\lastskip
     \vskip-\labelskip
     \next
     \vskip\labelskip
   \else
     \next
   \fi
}


\def\m@kelabel#1{label@#1}% this makes the controlseqence from the name

\def\labeldef#1#2{% #1 is name #2 is page section subsection figure table code
\expandafter\gdef\csname\m@kelabel{#1}\endcsname{#2}}%

\def\@setref#1#2#3{%
  \ifx#1\relax
   \message{Undefined reference: #3}
  \else
   \expandafter#2#1%
  \fi}

\def\@iofvii#1#2#3#4#5#6#7{#1}
\def\@iiofvii#1#2#3#4#5#6#7{#2}
\def\@iiiofvii#1#2#3#4#5#6#7{#3}
\def\@ivofvii#1#2#3#4#5#6#7{#4}
\def\@vofvii#1#2#3#4#5#6#7{#5}
\def\@viofvii#1#2#3#4#5#6#7{#6}
\def\@viiofvii#1#2#3#4#5#6#7{#7}


\def\pageref#1{\@readauxfile
  \pagelnk{\expandafter\@setref\csname\m@kelabel{#1}\endcsname\@iofvii{#1}}}
\def\secref#1{\@readauxfile
  \sectionlink
    {\expandafter\@setref\csname\m@kelabel{#1}\endcsname\@iiofvii{#1}}%
    {\expandafter\@setref\csname\m@kelabel{#1}\endcsname\@iiiofvii{#1}}%
}
\def\figref#1{\@readauxfile
\figlink{\expandafter\@setref\csname\m@kelabel{#1}\endcsname\@ivofvii{#1}}}
\def\tabref#1{\@readauxfile
\tablink{\expandafter\@setref\csname\m@kelabel{#1}\endcsname\@vofvii{#1}}}
\def\cdref#1{\@readauxfile
\expandafter\@setref\csname\m@kelabel{#1}\endcsname\@viofvii{#1}}
\def\enumref#1{\@readauxfile
\enumlink{\expandafter\@setref\csname\m@kelabel{#1}\endcsname\@viiofvii{#1}}}

%%
%%  Links and Destinations
%%

\ifbook
  \def\pdflinkcolor{0 0 0} % the RGB values for hyperlink color
  \def\linkcolor{\Black}
  \def\pdflink#1#2{#1}
  \def\pdfnote#1.{#1}
  \message{No PDF Links}
\else
  \def\pdflinkcolor{0 0 1} % the RGB values for hyperlink color
  \def\linkcolor{\Blue}
  \message{PDF Links are blue}
\fi

\def\@esphack{\relax\ifhmode\spacefactor\@savsf
     {}\ifdim \@savsk >\z@ \ignorespaces 
  \fi \fi}

%
% Code destinations: by number 
%

%The number is set be \M and \N and stored as \secno
%The destination is set in \X when we go into C mode
\def\codedest{%
\ifbook\else
\ifhint\HINTdest num \thecode top\relax
\else\ifpdf\pdfdest num \thecode fith\relax
\else\relax
\fi\fi\fi
}
\def\coderef#1{% used in \X
\ifbook${}_{#1}$\else
\ifhint\HINTstartlink goto num #1\lower 1.5pt\hbox{\sevenrm #1}\,\HINTendlink
\else\ifpdf{\eightrm\pdfnote#1.}
\else${}_{#1}$\fi\fi\fi
}

\def\codenote#1.{% replaces \pdfnote
  \setbox0=\hbox{\toksA={#1.}\toksB={}\maketoks}\the\toksA}
% define the following like for pdf
\def\firstsecno#1.{\setbox0=\hbox{\toksA={#1.}\toksB={}%
    \def\makenote{\addtokens\toksB{\the\toksC}\def\makenote{\toksD={}
      \toksC={}\let\space\empty}\makenote}\maketoks}}
\def\addtokens#1#2{\edef\addtoks{\noexpand#1={\the#1#2}}\addtoks}
\def\poptoks#1#2|ENDTOKS|{\let\first=#1\toksD={#1}%
  \ifcat\noexpand\first0\countB=`#1\else\countB=0\fi\toksA={#2}}
\def\maketoks{\expandafter\poptoks\the\toksA|ENDTOKS|%
  \ifnum\countB>`9 \countB=0 \fi
  \ifnum\countB<`0
    \ifnum0=\countC\else\makenote\fi
    \ifx\first.\let\next=\maketoksdone\else
        \let\next=\maketoks
        \addtokens\toksB{\the\toksD}
        \ifx\first,\addtokens\toksB{\space}\fi
    \fi
  \else \addtokens\toksC{\the\toksD}\global\countC=1\let\next=\maketoks
  \fi
  \next
}
\def\makenote{\addtokens\toksB
    {\noexpand\codelink{\the\toksC}{\romannumeral\the\toksC}}\toksC={}\global\countC=0}
\def\maketoksdone{\edef\st{\global\noexpand\toksA={\the\toksB}}\st}

\def\codelink#1#2{
\ifbook#1\else
\ifhint\HINTstartlink goto num #1 #1\HINTendlink
\else\ifpdf\pdflink{#1}{#2}%
\else#1%
\fi\fi\fi}

%
% new destinations and links: by name 
%

\def\newdest#1{% used to make a new destination
%\message{New destination #1}%
\ifbook
\else\ifpdf
  \pdfdest name {#1} fith%\message{Defining pdf label #1}%
\else\ifhint
  \HINTdest name {#1}%\message{Defining HINT label #1}%
\fi\fi\fi}

\def\newlink#1#2{%
  \ifbook #2\else
  \ifpdf
  \pdfstartlink goto name {#1}\linkcolor#2\Black\pdfendlink
  \else\ifhint
  \HINTstartlink goto name {#1}#2\HINTendlink
  \else
  #2%
  \fi\fi\fi
}


%SC.-\the\plaincount for plain sections
%SC.\thesection for regular sections, with subsections and letters for the appendix

\def\tocref#1#2#3#4{% depth, title, page, label used in tocline
  \ifbook#3\else
  \ifpdf
  \pdfstartlink goto name {SC.#4}\relax \linkcolor#3\Black\pdfendlink
   \ifnum#1=0\pdfoutline goto name {SC.#4} {#2}\fi
  \else\ifhint
    \HINTstartlink goto name {SC.#4}\relax $\rightarrow$\HINTendlink
    \HINToutline goto name {SC.#4} depth #1 {#2}%
  \else
  #3%
  \fi\fi\fi
}%

%
% Page references from the index
%

% \index{keyword} -> \@wrindex{file}{keyword} writes the nectessary information
% makeindex produces the index and pages are enclosed in \hyperref{pagenumber}
% Problems:
%   pagenumbers may be roman numerals in the frontmatter
%   hyperpage has to cope with 6 different formats
%   \hyperpage{50} a single page number
%   \hyperpage{51, 52} two numbers
%   \hyperpage{53--59} a page range
%   \hyperpage{xi} a roman numeral
%   \hyperpage{xi, xii} two numbers
%   \hyperpage{xi--xx} a page range


% Writing the index file: in HINT replacing the page number by the index count
\newcount\indexcount
\def\theindexcount{\the\indexcount}

\ifhint
  \def\@wrindex#1#2{% file, keyword
     \global\advance\indexcount by 1\indexdest
     \immediate\write#1{\string
        \indexentry{#2:hyperpage}{\the\indexcount}}\endgroup
     \if@nobreak \ifvmode\@nobreak\fi\fi\@esphack}
 \else
   \def\@wrindex#1#2{%
      \xdef\gtempa{\write#1{\string
      \indexentry{#2:hyperpage}{\noexpand\thepageno}}}\endgroup\gtempa
      \if@nobreak \ifvmode\@nobreak\fi\fi\@esphack}
\fi


% for HINT and PDF we used hyperrange and hypercomma to split the
% arguments into index/page numbers and then use pagelink
\ifbook
  \def\hyperpage#1{#1}\else
\ifhint
  \def\hyperpage#1{\hyperrange#1----+\relax}
\else\ifpdf
   \def\hyperpage#1{\hyperrange#1----+\relax}
\else 
   \def\hyperpage#1{#1}
\fi\fi\fi

%\def\indexlink#1{#1}

\def\hyperrange#1--#2--#3+{%
  \ifx\empty#2\empty%
    \hypercomma#1, ,+%
  \else\ifhint
    \pagelnk{#1}--\pagelink{#2}{$\leftarrow$}%
  \else
    \pagelnk{#1}--\pagelnk{#2}%
  \fi\fi}

\def\hypercomma#1, #2,#3+{%
  \ifx\empty#2\empty%
    \pagelnk{#1}%
  \else\ifhint%
    \pagelink{#1}{$\longrightarrow$}%
  \else
    \pagelnk{#1}%  
% I omit the second page number  \ifpdf, \pagelnk{#2}\fi
  \fi\fi
}


%
% Labels
%

% section references from \secref use \sectionlink
% for plainsections, we display the label name because there
%is no section number.

\def\sectionlink#1#2{% \message{Sectionlink #1 : #2}%
  \ifbook#1\else
  \ifpdf
  \pdfstartlink goto name {SC.#2} \linkcolor#1\Black\pdfendlink
  \else\ifhint
  \HINTstartlink goto name {SC.#2}#1\HINTendlink
  \else
  #1%
  \fi\fi\fi
}


% page references from \label and \pageref use \pagelink as before
\def\indexdest{\newdest{page.\the\indexcount}}
\def\pagelnk#1{%
  \ifhint
    \pagelink{#1}{$\rightarrow$}%
  \else
    \pagelink{#1}{#1}%
  \fi
}
\def\pagelink#1#2{\newlink{page.#1}{#2}}
% For hint the destination is generated with each \index and \label
% Because of the roman numerals, for pdf files
% the pdf destination is generated in the header using \pagelabel


\ifbook\let\pagelabel=\relax\else
\ifpdf
\def\pagelabel{%\message{pagelabel=\thepageno}
\pdfdest name {page.\thepageno} fitv\relax}
\else
\let\pagelabel=\relax
\fi\fi

% references to Figures, Tables, Enumerations
\def\figdest{\newdest{FI.\the\figcount}}
\def\figlink#1{\newlink{FI.#1}{#1}}
\def\tabdest{\newdest{TA.\the\tabcount}}
\def\tablink#1{\newlink{TA.#1}{#1}}
\def\enumdest{\newdest{EN.\the\enum}}
\def\enumlink#1{\newlink{EN.#1}{#1}}

%
% References to the bibliography
%
\def\citedest#1{\newdest{CI.#1}}
\def\citelink#1{%\message{Citelink #1}%
  \newlink{CI.#1}{#1}}

\def\bblhook{%defined before reading the .bbl file
  \def\biblabelprint##1{%
   \citedest{##1}%
   \noindent
   \hbox to \biblabelwidth{%
      \biblabelprecontents
      \biblabelcontents{##1}%
      \biblabelpostcontents
   }%
   \kern\biblabelextraspace
  }%
}
\def\@onecitation#1\@@{%called for each citation
   \if@notfirstcitation
      \printbetweencitations
   \fi
   %
   \expandafter \ifx \csname\@citelabel{#1}\endcsname \relax
      \if@citewarning
         \message{\@linenumber Undefined citation `#1'.}%
      \fi
      % Give it a dummy definition:
      \expandafter\gdef\csname\@citelabel{#1}\endcsname{%
         {\tt
            \escapechar = -1
            \nobreak\hskip0pt
            \expandafter\string\csname#1\endcsname
            \nobreak\hskip0pt
         }%
      }%
   \fi
   % Now produce the text, whether it was undefined or not.
   \citelink{\csname\@citelabel{#1}\endcsname}%
   \@notfirstcitationtrue
}%  
%%
%% Images
%%

\ifbook
\def\includefig#1{\leavevmode\hbox{\immediate\pdfximage{image/#1.pdf}\pdfrefximage\pdflastximage}}
\else\ifpdf
\def\includefig#1{\leavevmode\hbox{\immediate\pdfximage{image/#1.pdf}\pdfrefximage\pdflastximage}}
\else\ifhint
 \def\includefig#1{\HINTimage=image/#1.png\relax}
\else
  \input epsf.tex
  \def\includefig#1{\epsfbox{image/#1.eps}}% this is a \leavevmode\hbox{...}
\fi\fi\fi

\ifhint\eject\fi% hitex will remove empty pages that were introduced by cwebmac
\makeatother
